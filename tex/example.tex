\documentclass[10pt,oneside]{scrreprt}
\usepackage{amsthm,amsmath,amssymb,mathtools}
\usepackage{packages/rooms}
\usepackage{packages/definitions}

%%%%%%%%%%%%%%%%%%%%%%%%%%%%%%%%%%%%%%%%%%%%%%%%%%
%%%%%%%%%%%%%%%%%% Definitions %%%%%%%%%%%%%%%%%%%
%%%%%%%%%%%%%%%%%%%%%%%%%%%%%%%%%%%%(fold)

%Usage:
% \newdef [number of arguments]
%         {name for makro}
%         {content for makro}
%         {alternative content used for glossary entry}
%         {name of the defined term}
%         {glossary section (a letter or Sym for symbols)
%         {the actual definition}

\newdef{V}{V}{}{space}{V}{%
finite-dimensional complex inner-product space}

\newdef{T}{T}{}{Hermitian operator}{T}{%
Hermitian operator on \V.
}

\newdef[2]{innpro}{\langle #1, #2 \rangle}{\langle u, v \rangle}{inner product}{Sym}{%
the inner product of two vectors $u$ and $v$.
}

\newdef{herm}{\text{Hermitian}}{}{Hermitian}{H}{%
a operator $\T$ on a inner product space $\V$ is called Hermitian if $\innpro{T u}{v} = \innpro{u}{T v}$ for all vectors $u$ and $v$ in $\V$.}


\newdef{orth}{^{\bot}}{U^{\bot}}{orthogonal complement}{Sym}{%
the orthogonal complement for a subspace $U$ of $\V$ defined as the subspace
$\{ v \in \V\innpro{v}{u}=0 \text{ for all } u \in U\}$
}
%(end)

%%%%%%%%%%%%%%%%%%%%%%%%%%%%%%%%%%%%%%%%%%%%%%%%%%
%%%%%%%%%% Structure of the assertions %%%%%%%%%%%
%%%%%%%%%%%%%%%%%%%%%%%%%%%%%%%%%%%%(fold)
 
\begin{newstructure}{canonical-forms}
    \newstate{completeness}{
      \newstate{T-invariance}{}
      \newstate{contains-characteristic-vector}{}
    }
    \newstate{orthonormal-basis}{
      \newstate{lemma}{}
    }
\end{newstructure}

%(end)

%%%%%%%%%%%%%%%%%%%%%%%%%%%%%%%%%%%%%%%%%%%%%%%%%%
%%%%%%%%%% Definition of the assertions %%%%%%%%%%
%%%%%%%%%%%%%%%%%%%%%%%%%%%%%%%%%%%%(fold)

\defstate[\citepdf{Halmos}]{canonical-forms}{Theorem on canonical forms for Hermitian operators and matrices}{%
Let $\V$ be a finite-dimensional complex inner-product space, and let $\T$ be a \hermname operator on $\V$. Then $\V$ has an orthonormal basis of charateristic vectors of $\T$.
}

\defstate{completeness}{}{%
Let $U$ be the subspace $\V$ spanned by all characteristic vectors of $\T$. Then $U=\V$, i.e.m the characteristic vectors of $\T$ span the whole space.
}

\defstate{orthonormal-basis}{}{%
$U$ has an orthonormal basis of characteristic vectors of $\T$.
}

\defstate{T-invariance}{}{%
  $U\orth$ is $\T$-invariant.
}

\defstate{contains-characteristic-vector}{}{%
Every nonzero $\T$-invariant subspace of $\V$ contains a characteristic vector of $\T$.
}

\defstate{lemma}{Lemma}{%
Let $\T$ be a Hermitian operator. Then
\begin{enumerate}
  \item All the (complex) characteristic values of $\T$ are real.
  \item Characteristic vectors of $\T$ that belong to distinct characteristic values are orthogonal.
\end{enumerate}
}

%(end)


\begin{document}
%  \pagenumbering{arabic}
The  following  example is taken from \citepdf{Leron(1983)}.
\section*{Level 1}
\setroom{canonical-forms}
\roomtree{
  \subroom{completeness}
  \subroom{orthonormal-basis}
}
 
\begin{proof} 
Cleary the two assertions \refstate{completeness} and \refstate{orthonormal-basis} 
yield the conclusion of Theorem \refstate{canonical-forms}.
\end{proof}
  
\elevator
From here on the proof branches to two independent subproofs, 
rooted at \refstate{completeness} and \refstate{orthonormal-basis}. 
Note that while the presentation proceeds strictly top-down, 
it does not force a top-down \emph{reading} of the proof. 
Some readers may prefer, for example, 
to read through the first branch (\refstate{completeness}) all the way to the bottom, 
then return to Level 2 and start on the second branch (\refstate{orthonormal-basis}).


\setroom{completeness}
\section*{Level 2}
\roomtree{
  \subroom{T-invariance}
  \subroom{contains-characteristic-vector}
}
\begin{proof}
To prove $U=\V$ we prove the equivalent statement $U\orth = \left\{ 0 \right\}$. 
This in turn will follow from \refstate{T-invariance} and \refstate{contains-characteristic-vector}. 
Since $U\orth$ cannot contain a characteristic vector of $\T$ 
(this would contradict $U \cap U\orth = \left\{ 0 \right\}$), 
we must have $U\orth = \left\{ 0 \right\}$, hence $U=\V$.
\end{proof}
  

\setroom{orthonormal-basis}
\roomtree{
  \subroom{lemma}
}
\begin{proof}
Let $\lambda_1, \ldots \lambda_k$ be distinct characteristic values of $\T$, 
and let $U_1, \ldots , U_k$ be there respective characteristic spaces:
\[
  U_i = \left\{ v \in \V \;|\; T v = \lambda_i v  \right\}.
\]
By the Gram-Schmidt orthogonalization process 
we can construct an orthonormal basis $B_i$ to each subspace $U_i$, 
and it turns out that without any more car on our part, 
the set $B = B_1 \cup \cdots \cup B_k$ already forms an orthonormal basis for $\V$. 
This is due to the important lemma proved in \refstate{lemma}: 

Characteristic vectors of $\T$ that belong to distinct characteristic values are orthogonal. 
Thus if $i \neq j$, $B_i$ and $B_j$ are orthogonal, 
hence $B$ is indeed an orthonormal basis for $U$. 
Since $B$ consists of characteristic vectors of $\T$, 
it satisfies all the requirements of Level \refstate{orthonormal-basis}
\end{proof}

\elevator
We leave out the proofs of \refstate{T-invariance}, \refstate{contains-characteristic-vector} and \refstate{lemma}, 
since Level 3 is the bottom level and the proofs appear in it in their standard (linear) form (see \citepdf{hoffman}{pp. 312-313}). 
Note, however, that it is only in \refstate{T-invariance} and \refstate{lemma} 
that the main hypothesis of the theorem (namely that $\T$ is Hermitian) finally enters the proof.
In contrast, \refstate{contains-characteristic-vector} is true for operators on complex spaces in general.
  
  
\roomservice
\begin{treedef}
  \rootnode{V}
  \rootnode{T}
  \node{herm}
  \rootnode{orth}
  % \subnode{}
\end{treedef}

\bibliography{example}
\bibliographystyle{alpha}
\end{document}