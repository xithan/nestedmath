%\newglossary[sla]{deflist}{sya}{sxa}{Definitionen}
\edef\currentletter{}  % Achtung: heftige Nebeneffekte. Der Inhalt von currentletter wird in allen newentry definition verwendet. 
 \newglossary[slb]{offen}{syb}{sxb}{Offene Statements}
 
%\newmdenv[nobreak=true,framemethod=default,innerleftmargin=0.5ex,rightline=false,innerrightmargin=0.5ex]{defframe}
% \newglossary[slb]{sb}{syb}{sxb}{Surgery Definitions}
% \newglossary[slc]{basics}{syc}{sxc}{Basics}
% \newglossary[sld]{ab}{syd}{sxd}{Algebra Basics (Emergency service - Level 3)}
% \newglossary[sle]{la}{sye}{sxe}{Local Algebra Basics (Emergency service - Level 3)}
% \newglossary[slf]{sgn}{syf}{sxf}{Signatures (Emergency service - Level 3)}
% \newglossary[slg]{sp}{syg}{sxf}{Spectra (Emergency service - Level 3)}
% \newglossary[slh]{delta}{syh}{sxh}{Delta sets (Emergency service - Level 3)}
% \newglossary{deflist}{sya}{sxa}{Definitionen}
% \newglossary{sb}{syb}{sxb}{Surgery Definitions}
% \newglossary{basics}{syc}{sxc}{Basics}
% \newglossary{ab}{syd}{sxd}{Algebra Basics (Emergency service - Level 3)}
% \newglossary{la}{sye}{sxe}{Local Algebra Basics (Emergency service - Level 3)}
% \newglossary{sgn}{syf}{sxf}{Signatures (Emergency service - Level 3)}
% \newglossary{sp}{syg}{sxf}{Spectra (Emergency service - Level 3)}
% \newglossary{delta}{syh}{sxh}{Delta sets (Emergency service - Level 3)}

%Den Punkt am Ende jeder Beschreibung deaktivieren
\renewcommand*{\glspostdescription}{}


%\newcommand{\dist}{\hspace{0.5cm}}
\newcommand{\dist}{\kern0.5cm }

%Glossar-Befehle anschalten
%\makeglossaries
\newglossarystyle{defliststyle}{%
\glossarystyle{index}% base this style on the list style
\renewenvironment{theglossary}%
     {\begin{itemize}}%
    {\end{itemize}}
\renewcommand*{\glossaryentryfield}[5]{%
\item % bullet point
[\glstarget{##1}{\ensuremath{##2}}]% the entry name
%\space (##4)% the symbol in brackets
\space ##3% the description
\space (Def. p.\pageref{def##1})%
\space [##5]% the number list in square brackets
}%
}%

\newglossarystyle{widedefliststyle}{%
\glossarystyle{index}% base this style on the list style
\renewenvironment{theglossary}%
     {}%
    {}
\renewcommand*{\glossaryentryfield}[5]{%
%
\begin{defn}%
\noindent\nurrand{\glstarget{##1}{\ensuremath{##2}}}% the entry name
%\space (##4)% the symbol in brackets
\space ##3% the description
\space [##5]% the number list in square brackets
\end{defn}%
}%
}%

\newcommand{\newdef}[4][0]{%
\ifthenelse{\equal{#1}{0}}%
{\newglossaryentry{#2}{
	name=\ensuremath{#3},
	description={#4},
	sort=#2,
	type=deflist
	}}%
	{\newglossaryentry{#2}{
	name=\ensuremath{#3},
	description={#4},
	sort=#1,
	type=deflist
	}}
}

\newcommand{\linktest}{gls}



% \newentry[Anzahl Parameter][Optionale Belegung, leer geht nicht. Muss ein Leerzeichen enthalten, wenn der Standard leer sein soll]{Befehlsname}{Ausgabe fuer den Befehl}{Ausgabe im Glossar}{Ausfuerhliche Bezeichnung}{Definition}
\newcounter{para}
\setcounter{para}{1}

%  [Anzahl Parameter][Optionaler Paramter]{Makroname}{Makroinhalt}{Bezeichnung im Glossar}{Kurzdefinition}{Name}{Glossar Buchstabe}{Ausfuehrliche Definition}

\def\mydefton#1{\expandafter\newcommand\csname #1ton\endcsname{}}
\def\tonfor#1{\ifx#1\tonfor\else\mydefton#1\expandafter\tonfor\fi}
\tonfor ABCDEFGHIJKLMNOPQRSTUVWXYZ\tonfor
\newcommand{\Symton}{}
\newcommand{\ZZton}{} % Das was wahrscheinlich nicht aufgefuhert werden soll


% Ein extra parameter fuer das Symbol im Table of Notation
\newcommand{\mymagicentry}[1][0]{\setcounter{para}{#1}\mymagicentryy}
\newcommand{\mymagicentryeq}[1][0]{\setcounter{para}{#1}\mymagicentryyeq}  
\newcommand{\mymagicentrymap}[1][0]{\setcounter{para}{#1}\mymagicentryymap}
\newcommand{\mymagicentryns}[1][0]{\setcounter{para}{#1}\mymagicentryyns}
\newcommand{\mymagiconlyentry}[1][0]{\setcounter{para}{#1}\mymagiconlyentryy}

\newcommand{\magicentry}[1][0]{\setcounter{para}{#1}\magicentryy}
\newcommand{\magicentryeq}[1][0]{\setcounter{para}{#1}\magicentryyeq}  
\newcommand{\magicentrymap}[1][0]{\setcounter{para}{#1}\magicentryymap}
\newcommand{\magicentryns}[1][0]{\setcounter{para}{#1}\magicentryyns}
% Erstellt den verlinkbaren Eintrag, aber nicht den Verweisbefehl dazu
\newcommand{\magiconlyentry}[1][0]{\setcounter{para}{#1}\magiconlyentryy}

%  [Anzahl Parameter][Optionaler Paramter]{Makroname}{Makroinhalt}{Bezeichnung im Glossar}{Kurzdefinition}{Name}{Ausfuehrliche Definition}
\newcommand{\fullentry}[1][0]{\setcounter{para}{#1}\fullentryy}
\newcommand{\fullentryeq}[1][0]{\setcounter{para}{#1}\fullentryyeq}
\newcommand{\fullentrymap}[1][0]{\setcounter{para}{#1}\fullentryymap}
\newcommand{\fullentryns}[1][0]{\setcounter{para}{#1}\fullentryyns}

\newcommand{\newentry}[1][0]{\setcounter{para}{#1}\newentryy}
\newcommand{\newentryeq}[1][0]{\setcounter{para}{#1}\newentryyeq}
\newcommand{\newentryns}[1][0]{\setcounter{para}{#1}\newentryyns}
\newcommand{\newentrymap}[1][0]{\setcounter{para}{#1}\newentryymap}  
% Erstellt den verlinkbaren Eintrag, aber nicht den Verweisbefehl dazu
\newcommand{\onlyentry}[1][0]{\setcounter{para}{#1}\onlyentryy}


\newcommand{\createmacro}[3]{%
	\ifx\\#1\\ % empty  also keine optionalen Paramter
      \expandafter\providecommand\csname#2\endcsname[\value{para}]{\mathentrylink{#2}{#3}}     % Macro mit Link zur  Definition 
      \expandafter\providecommand\csname r#2\endcsname[\value{para}]{\rand{\mathentrylink{#2}{#3}}} % Macro mit Link am Rand zur Definition
          \expandafter\providecommand\csname#2link\endcsname[1]{\entrylink{#2}{##1}}
      
   \else
		\expandafter\providecommand\csname#2\endcsname[\value{para}][{#1}]{\mathentrylink{#2}{#3}}
		\expandafter\providecommand\csname r#2\endcsname[\value{para}][{#1}]{\rand{\mathentrylink{#2}{#3}}}
	\fi
}

\newcommand{\curdim}{n}
\newcommand{\dimnpone}{\renewcommand{\curdim}{n+1}}
\newcommand{\dimnmone}{\renewcommand{\curdim}{n-1}}
\newcommand{\setdim}[1]{\renewcommand{\curdim}{#1}}
\newcounter{textparas}



\newcommand{\createtextmacros}[5][]{%
% Anzahl Parameter steht in \value{para}
% 1 - Quelle und Ziel einer Abbildung
% 2 - Defaultbelegung fuer optionalen Paramter
% 3 - Name des Macros
% 4 - Alternative/original definition des Makros (muss vorher festgestellt werden, welche verwendet wird)
% 5 - Name des definierten Begriffs

% macro+name = ausfuehrlicher Name fuer Definiton
% macro+full = name und symbol
  
   \ifx\\#2\\ % Kein optionaler Parameter fuer den Bezeichner. Es gibt einen Optionalen Parameter, fuer was nach dem Befehl passieren soll
      %\setcounter{textparas}{\value{para}}
      %\stepcounter{textparas}
      \expandafter\providecommand\csname #3name\endcsname[1][]{\entrylink{#3}{#5}##1\xspace} % nur der Name der Definition, optionaler Parameter fuer Satzzeichen
      \expandafter\providecommand\csname #3namedim\endcsname[1][n]{\entrylink{#3}{$##1$-dimensional #5}\xspace}
      \expandafter\providecommand\csname #3full\endcsname[1][]{\entrylink{#3}{#5 \ensuremath{#4}#1}##1\xspace}
      \expandafter\providecommand\csname #3fulldim\endcsname[1][n]{\entrylink{#3}{$##1$-#5 \ensuremath{#4}#1}\xspace}
      \expandafter\providecommand\csname #3defaultsym\endcsname{\ensuremath{#4}}
      \ifx\\#1\\
      \else  % Es wird eine Abbildung definiert
         \expandafter\providecommand\csname #3map\endcsname{\entrylink{#3}{\ensuremath{#4}}#1}
      \fi
   \else
      \expandafter\providecommand\csname #3defaultsym\endcsname[1][#2]{\ensuremath{#4}}
      \expandafter\providecommand\csname #3full\endcsname[\value{para}][{#2}]{\entrylink{#3}{#5 \ensuremath{#4}#1}\xspace} % Name und Bezeichner fuer Definitiion
      \expandafter\providecommand\csname #3name\endcsname[\value{para}][{#2}]{\entrylink{#3}{#5}\xspace}
   \fi
}


\newcommand{\mymagicentryyns}[8][]{%
% Parameter: [Defaultbelegung fuer optionalen Parameter fuer das zu erzeugende Macro]
%              {Name des Macros} {Inhalt des Macros} {Alternativer Inhalt des Macros Definition}{Glossar Buchstabe} {Ausfuerhliche Bezeichnung} {Definitionstext}
\fullentryyns[#1]{#2}{#3}{#4}{#5}{#8}
\addentrytoton{#7}{#6}{#2}
}

\newcommand{\mymagicentryy}[8][]{%
% 1 - Defaultbelegung fuer optionalen Parameter
% 2 - Name fuer Macro
% 3 - Symbol fuer Macro (fuer Gebrauch im Text)
% 4 - Alternatives Symbol fuer Glossar
% 5 - Name des definierten Begriffs
% 6 - Alternatives Symbol fuer Table of Notation
% 7 - Buchstabe fuer Table of Notation
% 8 - Die Definition 
\fullentryy[#1]{#2}{#3}{#4}{#5}{#8}
\addmyentrytoton{#7}{#6}{#2}
}

\newcommand{\mymagicentryyeq}[9][]{%
% 1 - Defaultbelegung fuer optionalen Parameter
% 2 - Name fuer Macro
% 3 - Symbol fuer Macro (fuer Gebrauch im Text)
% 4 - Alternatives Symbol fuer Glossar
% 5 - Definition der Gleichung (Anfang im Glossar)
% 6 - Name des definierten Begriffs
% 7 - Alternatives Symbol fuer Table of Notation
% 8 - Buchstabe fuer Table of Notation
% 9 - Die Definition 
\fullentryyeq[#1]{#2}{#3}{#4}{#5}{#6}{#9}
\addmyentrytoton{#8}{#7}{#2}
}

\newcommand{\mymagicentryymap}[9][]{%
% 1 - Defaultbelegung fuer optionalen Parameter
% 2 - Name fuer Macro
% 3 - Symbol fuer Macro (fuer Gebrauch im Text)
% 4 - Alternatives Symbol fuer Glossar
% 5 - Definition der Abbildung (Anfang im Glossar)
% 6 - Name des definierten Begriffs
% 7 - Alternatives Symbol fuer Table of Notation
% 8 - Buchstabe fuer Table of Notation
% 9 - Die Definition 
\fullentryymap[{#1}]{#2}{#3}{#4}{#5}{#6}{#9}
\addmyentrytoton{#8}{#7}{#2}
}
% 2 - Name fuer Macro
% 3 - Definitoin
% 4 - Name
% 5 - Alternatives Symbol fuer Table of Notation
% 6 - Buchstabe fuer Table of Notation
% 7 - Die Defintion
\newcommand{\mymagiconlyentryy}[7][]{%
    \onlyentryy[#1]{#2}{#3}{#7}

    \expandafter\providecommand\csname #2defaultsym\endcsname{\ensuremath{#3}}
    \expandafter\providecommand\csname #2name\endcsname{\entrylink{#2}{#4}\xspace}
    \addmyentrytoton{#6}{#5}{#2}
}


\newcommand{\magicentryyns}[7][]{%
% Parameter: [Defaultbelegung fuer optionalen Parameter fuer das zu erzeugende Macro]
%              {Name des Macros} {Inhalt des Macros} {Alternativer Inhalt des Macros Definition}{Glossar Buchstabe} {Ausfuerhliche Bezeichnung} {Definitionstext}
\fullentryyns[#1]{#2}{#3}{#4}{#5}{#7}
\addentrytoton{#6}{#2}
}

\newcommand{\magicentryy}[7][]{%
% 1 - Defaultbelegung fuer optionalen Parameter fuer das zu erzeugende Macro
% 2 - Name des Macros
% 3 - Symbol fuer Macro (die Definition des Macro)
% 4 - Alternatives Symbol fuer Glossar
%  {Ausfuerhliche Bezeichnung} 
% 6 - Die mathematische Definition
\fullentryy[#1]{#2}{#3}{#4}{#5}{#7}
\addentrytoton{#6}{#2}
}

\newcommand{\magicentryyeq}[8][]{%
% 1 - Defaultbelegung fuer optionalen Parameter
% 2 - Name fuer Macro
% 3 - Definition fuer Macro (fuer Gebrauch im Text)
% 4 - Alternative Defintion fuer Glossar
% 5 - Definition der Gleichung (Anfang im Glossar)
% 6 - Name des definierten Begriffs
% 7 - Glossar Buchstabe
% 8 - Die Definition 
\fullentryyeq[#1]{#2}{#3}{#4}{#5}{#6}{#8}
\addentrytoton{#7}{#2}
}

\newcommand{\magicentryymap}[8][]{%
% 1 - Defaultbelegung fuer optionalen Parameter
% 2 - Name fuer Macro
% 3 - Definition fuer Macro (fuer Gebrauch im Text)
% 4 - Alternative Defintion fuer Glossar
% 5 - Definition der Abbildung (Anfang im Glossar)
% 6 - Name des definierten Begriffs
% 7 - Glossar Buchstabe
% 8 - Die Definition 
\fullentryymap[{#1}]{#2}{#3}{#4}{#5}{#6}{#8}
\addentrytoton{#7}{#2}
}

% 2 - Name fuer Macro
% 3 - Definitoin
% 4 - Name
% 5 - Glossar Buchstabe
% 6 - Die Defintion
\newcommand{\magiconlyentryy}[6][]{%
    % \oneforall[#1]{#2}{%
    %   \ensuremath{#3}\dist
    %   #6%
    % }
    \onlyentryy[#1]{#2}{#3}{#6}
    \expandafter\providecommand\csname #2defaultsym\endcsname{\ensuremath{#3}}
    \expandafter\providecommand\csname #2name\endcsname{\entrylink{#2}{#4}\xspace}
    \addentrytoton{#5}{#2}
}

\newcommand{\fullentryyeq}[7][]{%
  \newentryyeq[#1]{#2}{#3}{#4}{#5}{#7}
   \ifx\\#4\\ % keine alternative Bezeichung fuer den Term
      \createtextmacros{#1}{#2}{#3}{#6}
	\else
      \createtextmacros{#1}{#2}{#4}{#6}
	\fi
}

\newcommand{\fullentryymap}[7][]{%
% 1 - Defaultbelegung fuer optionalen Parameter
% 2 - Name fuer Macro
% 3 - Definition fuer Macro (fuer Gebrauch im Text)
% 4 - Alternative Defintion fuer Glossar
% 5 - Definition der Abbildung (Anfang im Glossar)
% 6 - Name des definierten Begriffs
% 7 - Die Definition 
  \newentryymap[#1]{#2}{#3}{#4}{#5}{#7}
   \ifx\\#4\\ % keine alternative Bezeichung fuer den Term
      \createtextmacros[{#5}]{#1}{#2}{#3}{#6}
	\else
      \createtextmacros[{#5}]{#1}{#2}{#4}{#6}
	\fi
}%



\newcommand{\fullentryy}[6][]{%
% Parameter: [Defaultbelegung fuer optionalen Parameter fuer das zu erzeugende Macro]
%              {Name des Macros} {Inhalt des Macros} {Alternativer Inhalt des Macros Definition} {Ausfuerhliche Bezeichnung} {Definitionstext}
  \newentryy[#1]{#2}{#3}{#4}{#6}
   \ifx\\#4\\ % keine alternative Bezeichung fuer den Term
      \createtextmacros{#1}{#2}{#3}{#5}
	\else
      \createtextmacros{#1}{#2}{#4}{#5}
	\fi
}%

\newcommand{\fullentryyns}[6][]{%
% Parameter: [Defaultbelegung fuer optionalen Parameter fuer das zu erzeugende Macro]
%              {Name des Macros} {Inhalt des Macros} {Alternativer Inhalt des Macros Definition} {Ausfuerhliche Bezeichnung} {Definitionstext}
    \newentryyns[#1]{#2}{#3}{#4}{#6}
     \ifx\\#4\\ % keine alternative Bezeichung fuer den Term
        \createtextmacros{#1}{#2}{#3}{#5}
  	\else
        \createtextmacros{#1}{#2}{#4}{#5}
  	\fi
}%

\newcommand{\onlyentryy}[4][]{
    \inituml{#2}
    \oneforevenmore[#1]{#2}{\ensuremath{#3}}{\dist #4}
}

\newcommand{\newentryy}[5][]{%
  \inituml{#2}
	\ifx\\#4\\ % keine alternative Bezeichung fuer die Definition
    \oneforevenmore[#1]{#2}{\ensuremath{#3}}{\dist #5}
	\else
    \oneforevenmore[#1]{#2}{\ensuremath{#4}}{\dist #5}
	\fi
   \createmacro{#1}{#2}{#3}
}%
   
\newcommand{\newentryyns}[5][]{%
    \inituml{#2}
    \ifx\\#4\\ % keine alternative Bezeichung fuer die Definition
          \oneforevenmore[#1]{#2}{\ensuremath{#3}}{\kern 1ex #5}
    \else
          \oneforevenmore[#1]{#2}{\ensuremath{#4}}{\kern 1ex #5}
    \fi
    \createmacro{#1}{#2}{#3}
}%
      
\newcommand{\newentryymap}[6][]{%
    \inituml{#2}
		\ifx\\#4\\ % keine alternative Bezeichung fuer die Definition
      \oneforevenmore[#1]{#2}{\ensuremath{#3}#5}{\dist#6}
		\else
      \oneforevenmore[#1]{#2}{\ensuremath{#4}#5}{\dist#6}
		\fi
      \createmacro{#1}{#2}{#3}}%
      
\newcommand{\newentryyeq}[6][]{%
    \inituml{#2}
		\ifx\\#4\\ % keine alternative Bezeichung fuer die Definition
      \oneforevenmore[#1]{#2}{\ensuremath{#3}#5}{\dist#6}
		\else
        \oneforevenmore[#1]{#2}{\ensuremath{#4}#5}{\dist#6}
		\fi
      \createmacro{#1}{#2}{#3}}%

\newcommand{\umlinit}{}

\newcommand{\inituml}[1]{\expandafter\providecommand\csname uml#1\endcsname{}}


\mdfdefinestyle{treeboxstyle}{innerleftmargin=4pt,innerrightmargin=3pt,innerbottommargin=2pt,innertopmargin=2pt,align=center,nobreak,topline=true,rightline=true,leftline=true}
%linewidth=10pt,

%\usepackage{environ,}
\newsavebox\MyTempBox
\NewEnviron{defframe}{%
\savebox\MyTempBox{%
\begin{varwidth}{\linewidth}
\BODY%
\end{varwidth}}%
\begin{minipage}{\dimexpr\wd\MyTempBox+8pt\relax}
\begin{mdframed}[style=treeboxstyle,userdefinedwidth=\dimexpr\wd\MyTempBox+8pt\relax]
\BODY%
\end{mdframed}%
\end{minipage}}

% erzeugt das makro fuer die vollstaendige Glossardefinition
\newcommand{\oneforevenmore}[4][]{
    \ifx\\#1\\%empty     % wenn es keinen Defaultwert fuer einen optionalen Parameter gibt, muessen moegliche Parameter in der Definition bereits durch einen alternative parameterlose Definition ersetzt worden sein.
  			\expandafter\newcommand\csname defcontent#2x\currentletter\endcsname{#4}
        \expandafter\newcommand\csname defkey#2x\currentletter\endcsname{\begin{defframe}\strut#3\end{defframe}}
  			\expandafter\newcommand\csname def#2x\currentletter\endcsname{%
          %\setdeftarget{#2}%\target{gls\csname uml#2\endcsname x#2}%
          %\kern-0.5ex\fbox{#3}#4%
          %\kern-0.5ex \begin{minipage}{3cm}\begin{defframe}#3\end{defframe}\end{minipage}#4%
          \kern-0.5ex \begin{defframe}\strut#3\end{defframe}#4%
  			}%
  			% \expandafter\newcommand\csname hangdef#2x\currentletter\endcsname{%
  			%           %\setdeftarget{#2}%\target{gls\csname uml#2\endcsname x#2}%
  			%           \kern-0.5ex\fbox{#3}{\hangindent=0.5cm #4}%
  			%         }%
  	\else%
		    \expandafter\newcommand\csname defcontent#2x\currentletter\endcsname[\value{para}][#1]{#4}
        %\expandafter\newcommand\csname defkey#2x\currentletter\endcsname[\value{para}][#1]{\fbox{#3}}
        \expandafter\newcommand\csname defkey#2x\currentletter\endcsname[\value{para}][#1]{\begin{defframe}\strut#3\end{defframe}}
  			\expandafter\newcommand\csname def#2x\currentletter\endcsname[\value{para}][#1]{%
          %\setdeftarget{#2}%\target{gls\csname uml#2\endcsname x#2}%
          %\kern-0.5ex\fbox{#3}#4%
          \kern-0.5ex \begin{defframe}\strut#3\end{defframe}#4%
  			}%
        % \expandafter\newcommand\csname hangdef#2x\currentletter\endcsname[\value{para}][#1]{%
        %           %\setdeftarget{#2}%\target{gls\csname uml#2\endcsname x#2}%
        %           \kern-0.5ex\fbox{#3}{\hangindent=0.5cm #4}%
        % }%
  	\fi}



\newcommand{\define}[2][]{\noindent\csname def#2\endcsname{#1}\\[1ex]}%

% \newentry[Sortkey (if not defined = reference name)]{reference name}{Name of Entry}{Name of Glossary}{Description}
\newcommand{\newglsentry}[5][0]{%
\ifthenelse{\equal{#1}{0}}%
{\newglossaryentry{#2}{
	name=\ensuremath{#3},
	description={#5},
	sort=#2,
	type={#4}
	}}%
	{\newglossaryentry{#2}{
	name=\ensuremath{#3},
	description={#5},
	sort=#1,
	type={#4}
	}}
}

\makeatletter
\newcommand{\myglslabel}[1]{
    \@ifundefined{r@glls#1}{\label{glls#1}}{}
}
\makeatother

% Setzt target fuer Glossareintraege im Tree of Notations    
% \newcommand{\settontarget}[1]{%
%   \currentletter
%   gls\csname uml#1\endcsname x#1
%   \target{gls\csname uml#1\endcsname x#1}
%   \myglslabel{\csname uml#1\endcsname x#1}
% }    
    
\newcommand{\setdeftarget}[1]{%
    % 
    % \else  % aller erste definition
    \ifcsname alreadydef#1\endcsname%
    \else%
      \expandafter\xdef\csname alreadydef#1\endcsname{true}%
      %glsx#1
      \target{glsx#1}%
    \fi%
    %\currentletter
    %gls\csname uml#1\endcsname x#1
    \target{gls\csname uml#1\endcsname x#1}%
    \label{glls\csname uml#1\endcsname x#1}%
%    \@ifundefined{r@#1}{\target{gls#1}\label{glls#1}}{}
    %   \expandafter\gdef\csname alreadydef#1\endcsname{true}
    % \fi
    % \expandafter\ifx\csname uml#1\endcsname\umlinit
    % \else
    %   \target{gls\csname uml#1\endcsname x#1}
    % \fi
    % gls\csname uml#1\endcsname x#1
}
    


\newcommand{\addtoton}[2]{\expandafter\gappto\csname #1ton\endcsname{#2}}


\newcommand{\linkortarget}[1]{\protect{%
    \protect\ifcsname alreadydef#1\endcsname%
    %\@ifundefined{r@glls#1}
    \protect\getpagerefnumber{glls\csname uml#1\endcsname x#1}%
  \else%
    \protect\target{glsx#1}%
    %\protect\setdeftarget{#1}
    %\expandafter\gdef\csname alreadydef#1\endcsname{true}
%  \expandafter\ifx\csname uml#1\endcsname\umlinit % der erste Eintrag fuer diese Definition, nichts veraenern
  \fi}%
  %\expandafter\gappto\nextdeftree{#2}
}
% Add to Table of Notation
% Buchstabe, Symbol, Beschreibung

% \newcommand{\addentrytoton}[2]{%
%   \ifcsname alreadydef#2\endcsname
%     \addtoton{#1}{\csname #2defaultsym\endcsname & \csname #2name\endcsname &
%         \protect\getpagerefnumber{glls\csname uml#2\endcsname x#2} X \protect\entrylink{#2}{\getpagerefnumber{glls\csname uml#2\endcsname x#2}}\\}
%   \else
%     \addtoton{#1}{\csname #2defaultsym\endcsname & \csname #2name\endcsname & %\protect\setdeftarget{#2}
%     \\}
%     %\expandafter\gdef\csname alreadydef#1\endcsname{true}
%   \fi
% }

% \newcommand{\addmyentrytoton}[3]{%
%   \ifcsname alreadydef#3\endcsname
%     \addtoton{#1}{#2 & \csname #3name\endcsname &
%     \protect\getpagerefnumber{glls\csname uml#3\endcsname x#3} X \protect\entrylink{#3}{\getpagerefnumber{glls\csname uml#3\endcsname x#3}}\\}
%   \else
%     \addtoton{#1}{#2 & \csname #3name\endcsname & \protect\setdeftarget{#3}\\}
%     %\expandafter\gdef\csname alreadydef#1\endcsname{true}
%   \fi
% }




\newcommand{\addentrytoton}[2]{\addtoton{#1}{\csname #2defaultsym\endcsname & \csname #2name\endcsname & {\protect\linkortarget{#2}}\\}}
%\newcommand{\addentrynoreftoton}[2]{\addtoton{#1}{\csname #2defaultsym\endcsname & \csname #2name\endcsname &\\}}
\newcommand{\addmyentrytoton}[3]{\addtoton{#1}{#2 & \csname #3name\endcsname & \protect\linkortarget{#3}\\}}
% \newcommand{\addentrytoton}[2]{\addtoton{#1}{\csname #2defaultsym\endcsname & \csname #2name\endcsname & \protect\hyperlink{#2ton }{\getpagerefnumber{#2ton}}\\}}
% \newcommand{\addentrynoreftoton}[2]{\addtoton{#1}{\csname #2defaultsym\endcsname & \csname #2name\endcsname &\\}}
% \newcommand{\addmyentrytoton}[3]{\addtoton{#1}{#2 & \csname #3name\endcsname & \protect\hyperlink{#3ton}{\getpagerefnumber{#3ton}}\\}}




\newcommand{\deftarget}[1]{}

% alreadydef wird nicht gesetzt, damit beim Eintrag sowohl der globale als auch der ton target gesetzt wird
\newcommand{\settonuml}[1]{%
  % \ifcsname alreadydef#1\endcsname
  % \else
  %   \expandafter\xdef\csname 
  % \fi
  
  \expandafter\xdef\csname uml#1\endcsname{\currentletter} % neue Umleitung setzen 
  %\else
    %\expandafter\gdef\csname alreadydef#1\endcsname{true}
   % \expandafter\gdef\csname uml#1\endcsname{}  %umlinit
%  \expandafter\ifx\csname uml#1\endcsname\umlinit % der erste Eintrag fuer diese Definition, nichts veraenern
  %\fi}
}

\newcommand{\setuml}[1]{%
  %{\ifcsname alreadydef#1\endcsname%\relax
    \expandafter\xdef\csname uml#1\endcsname{\currentletter} % neue Umleitung setzen 
  %\else
  %  \expandafter\gdef\csname alreadydef#1\endcsname{true}
  %  \expandafter\gdef\csname uml#1\endcsname{}  %umlinit
%  \expandafter\ifx\csname uml#1\endcsname\umlinit % der erste Eintrag fuer diese Definition, nichts veraenern
 % \fi}
}

\newcommand{\addtotree}[1]{\expandafter\gappto\nextdeftree{#1}}

\newcommand{\deflocaluml}[2]{%
  \ifcsname alreadydef#1\endcsname
  \expandafter\edef\csname uml#1\endcsname{\currentletter} % neue Umleitung setzen 
  \else
    \expandafter\gdef\csname alreadydef#1\endcsname{true}
    \expandafter\gdef\csname uml#1\endcsname{}  %umlinit
%  \expandafter\ifx\csname uml#1\endcsname\umlinit % der erste Eintrag fuer diese Definition, nichts veraenern
  \fi
  \expandafter\gappto\nextdeftree{#2\restoreuml{#1}}
}

\connectnodes  % Der Baum soll Verbindungen bis zu den Knoten haben.

\newcommand{\treerootfor}[1]{\connectedtreeroot{\getdef{#1}}{\getdefkey{#1}}}
\newcommand{\treechildfor}[3][]{\connectedtreechild[#1]{#2}{\getdef{#3}}{\getdefkey{#3}}}


\newcommand{\rootnodei}[1]{\setuml{#1}\addtotree{\treerootfor{#1}}}
%\newcommand{\rootnodei}[1]{\setuml{#1}\addtotree{\treeroot{\hangindent=0.7cm \lipsum[1]}}}
%\newcommand{\groupnodei}[1]{\setuml{#1}\addtotree{\treechild[\grouplink]{0}{\getdef{#1}}}}
\newcommand{\groupnodei}[1]{\setuml{#1}\addtotree{\treechildfor[\grouplink]{0}{#1}}}

\newcommand{\specialnodei}[2][1]{\setuml{#2}\addtotree{\treechildfor[\corollarylink]{#1}{#2}}}
%\newcommand{\cornodei}[2][1]{\setuml{#2}\addtotree{\treechild[\corollarylink]{#1}{\getdef{#2}}}}   % Corollary-Node
\newcommand{\cornodei}[2][1]{\setuml{#2}\addtotree{\treechildfor[\corollarylink]{#1}{#2}}}   % Corollary-Node
\providecommand{\nodei}[2][1]{\setuml{#2}\addtotree{\treechildfor{#1}{#2}}}
\newcommand{\subnodei}[1]{\nodei[2]{#1}}
\newcommand{\subsubnodei}[1]{\nodei[3]{#1}}

%\newcommand{\localrootnodei}[1]{\setuml{#1}\addtotree{\treeroot{\getlocaldef{#1}}}\restorelastuml{#1}}
% getlocaldef ist nicht $(\noetig, weil get def sowieso localdef liefert falls exisitert
\newcommand{\localrootnodei}[1]{\setuml{#1}\addtotree{\treerootfor{#1}\restorelastuml{#1}}}
\newcommand{\localgroupnodei}[1]{\setuml{#1}\addtotree{\treechildfor[\grouplink]{0}{#1}\restorelastuml{#1}}}


\newcommand{\rootnode}[1]{\settonuml{#1}\treerootfor{#1}}
\newcommand{\groupnode}[1]{\settonuml{#1}\treechildfor[\grouplink]{0}{#1}}
\newcommand{\specialnode}[2][1]{\settonuml{#2}\treechildfor[\corollarylink]{#1}{#2}}
\newcommand{\cornode}[2][1]{\settonuml{#2}\treechildfor[\corollarylink]{#1}{#2}}   % Corollary-Node
\providecommand{\node}[2][1]{\settonuml{#2}\treechildfor{#1}{#2}}
\newcommand{\subnode}[1]{\node[2]{#1}}
\newcommand{\subsubnode}[1]{\node[3]{#1}}



% {\ifcsname def#1\currentletter\endcsname
% \treeroot{\csname def#1\currentletter\endcsname}
% \else
% \treeroot{\csname def#1\endcsname}
% \fi}
%{\csname def#1\endcsname{}}}
%\newcommand{\localrootnode}[1]{\treeroot{\getlocaldef{#1}}}
%\newcommand{\localgroupnode}[1]{\treechild[\grouplink]{0}{\getlocaldef{#1}}}
\newenvironment{treedefine}[1]{\newconnectedtree\treeroot{\getdef{#1}}}{\noargotree\relax}
\newenvironment{treedef}{\newconnectedtree}{\noargotree\relax}

 
\def\nextdeftree{\mydeftree}
%\newcommand{\thedeftree}{}
\newenvironment{treepredef}[1][\mydeftree]{%
\gdef\nextdeftree{#1}%
%\expandafter\gappto\nextdeftree{\relax}%
\gdef#1{\begin{treedef}\relax}}{\expandafter\gappto\nextdeftree{\end{treedef}}}

\newenvironment{graphdefine}[1]{\newconnectedtree\treeroot{{#1}\vspace{-\baselineskip}}}{\noargotree\relax}  % Veraltet. Wird jetzt mit groupnode gemacht


%%%%%%%%%%%%%%%%%%%%%%%%%%%%%%%%%%%%%%%%%%%%%%%%%%
%%%%%%%%%%%%%% Lokale Definitionen %%%%%%%%%%%%%%%
%%%%%%%%%%%%%%%%%%%%%%%%%%%%%%%%%%%%%%%%%%%%%%%%%%
% (fold)
\def\aster{*}
\def\setolduml#1 {\def\one{#1}%
\ifx\one\aster\let\go\relax
\else\expandafter\edef\csname uml#1\endcsname{\currentletter}\let\go\setolduml
\fi\go}


% Speichert die letzte Umleitung
\newcommand{\savelastuml}[1]{%
  \expandafter\xdef\csname lastuml#1\endcsname{\csname uml#1\endcsname}
  \expandafter\xdef\csname uml#1\endcsname{\currentletter}
}
% Stellt die letzte Umleitung wieder her
\newcommand{\restorelastuml}[1]{%
  \expandafter\xdef\csname uml#1\endcsname{\csname lastuml#1\endcsname}
}

\newcommand{\newlocalentry}[1][0]{\setcounter{para}{#1}\newlocalentryy}
\newcommand{\newlocalentryns}[1][0]{\setcounter{para}{#1}\newlocalentryyns}
\newcommand{\newlocalentrymap}[1][0]{\setcounter{para}{#1}\newlocalentryymap}
\newcommand{\newlocalentryeq}[1][0]{\setcounter{para}{#1}\newlocalentryyeq}

\newcommand{\newlocalentryy}[5][]{%
  \savelastuml{#2}
	\ifx\\#4\\ % keine alternative Bezeichung fuer die Definition
    \oneforevenmore[#1]{#2}{\ensuremath{#3}}{\dist #5}
	\else
    \oneforevenmore[#1]{#2}{\ensuremath{#4}}{\dist #5}
	\fi
   \createmacro{#1}{#2}{#3}
}%
\newcommand{\newlocalentryyns}[5][]{%
    \savelastuml{#2}
    \ifx\\#4\\ % keine alternative Bezeichung fuer die Definition
          \oneforevenmore[#1]{#2}{\ensuremath{#3}}{#5}
    \else
          \oneforevenmore[#1]{#2}{\ensuremath{#4}}{#5}
    \fi
    \createmacro{#1}{#2}{#3}
}%
\newcommand{\newlocalentryymap}[6][]{%
      \savelastuml{#2}
		\ifx\\#4\\ % keine alternative Bezeichung fuer die Definition
      \oneforevenmore[#1]{#2}{\ensuremath{#3}#5}{\dist#6}
		\else
      \oneforevenmore[#1]{#2}{\ensuremath{#4}#5}{\dist#6}
		\fi
      \createmacro{#1}{#2}{#3}
}%
\newcommand{\newlocalentryyeq}[6][]{%
    \savelastuml{#2}
		\ifx\\#4\\ % keine alternative Bezeichung fuer die Definition
    \oneforevenmore[#1]{#2}{\ensuremath{#3}#5}{\dist#6}
		\else
        \oneforevenmore[#1]{#2}{\ensuremath{#4}#5}{\dist#6}
		\fi
      \createmacro{#1}{#2}{#3}
}%


%\makeatletter
\gdef\restorecommands{\restoretemp\gdef\restoretemp{}}
\gdef\restoretemp{}
\newcommand{\newlocalcommand}[3]{%   {\Macroname}{\Macroname unter dem das alte Macro zwischengespeichert werden soll}{Macrocode}
   \let#2#1%
   \gdef#1{#3}%
   %\g@addto@macro\restoretemp{\let#1#2}%
   \gappto\restoretemp{\let#1#2}%
}
% \newcommand{\newlocalcommand}[2]{
%    \expandafter\edef\csname #1global\endcsname{\csname #1\endcsname}
%    \expandafter\def\csname #1\endcsname{#2}
%    \g@addto@macro\restorecommands{\expandafter\def\csname #1\endcsname{\csname #1global\endcsname}}%
% }
% \newcommand{\newlocalcommand}[2]{
%    \expandafter\let\csname #1global\endcsname#1
%    \expandafter\def#1{#2}
%    \g@addto@macro\restorecommands{\expandafter\expandafter\expandafter\let#1\csname #1global\endcsname}%
% }
%\makeatother

\newcommand{\uselocalentries}[1]{\edef\currentletter{\statealph{#1}}}
   
\newcommand{\getdef}[1]{%
\ifcsname def#1x\currentletter\endcsname
\setdeftarget{#1}\csname def#1x\currentletter\endcsname
\else\setdeftarget{#1}\csname def#1x\endcsname\fi}

\newcommand{\getdefkey}[1]{%
\ifcsname defkey#1x\currentletter\endcsname
\csname defkey#1x\currentletter\endcsname
\else\csname defkey#1x\endcsname\fi}



\newcommand{\gethangingdef}[1]{%
%\ifcsname hangdef#1x\currentletter\endcsname
\setdeftarget{#1}\kern-0.5ex\fbox{\csname defkey#1x\endcsname}{\expandafter\hangindent=0.7cm \csname defcontent#1x\endcsname
}
%\else\setdeftarget{#1}\csname hangdef#1x\endcsname\fi
}


\newcommand{\getlocaldef}[1]{\csname def#1x\currentletter\endcsname{}}

\newcommand{\csmacroproblem}[1]{\csname #1\endcsname{}}

\def\using{* }
\newcommand{\uses}[2]{%
\edef\currentletter{}
\expandafter\setolduml\using
\edef\currentletter{\statealph{#1}}
\setolduml#2 * 
\def\using{#2 * } 
}
% (end)